\documentclass[11pt, oneside]{article}   	% use "amsart" instead of "article" for AMSLaTeX format
\usepackage{geometry}                		% See geometry.pdf to learn the layout options. There are lots.
\geometry{letterpaper}                   		% ... or a4paper or a5paper or ... 
%\geometry{landscape}                		% Activate for rotated page geometry
\usepackage[parfill]{parskip}    			% Activate to begin paragraphs with an empty line rather than an indent
\usepackage{graphicx}				% Use pdf, png, jpg, or eps§ with pdflatex; use eps in DVI mode
								% TeX will automatically convert eps --> pdf in pdflatex		
\usepackage{amssymb}
\usepackage{hyperref}
\usepackage{amsmath}
\usepackage{listings}
\usepackage{mathtools}
\usepackage{enumerate}
\usepackage{tikz}
\usepackage{algorithm}   
\usepackage{algpseudocode} 
\usepackage{caption}  % Required for caption outside of figure
\newcommand{\ck}[1]{\textcolor{cyan}{CK: #1}}
\newcommand{\jc}[1]{\textcolor{orange}{JC: #1}}
\newcommand{\hm}[1]{\textcolor{blue}{HM: #1}}

\title{Homework 2 \\ CSC 277 / 477 \\ End-to-end Deep Learning \\ Fall 2024}
\author{Henry Yin - \texttt{hyin12@u.rochester.edu}}
\date{}					


\begin{document}

\maketitle

\begin{center}
    \textbf{Deadline:} 10/18/2024
\end{center}


\section*{Instructions}

Your homework solution must be typed and prepared in \LaTeX. It must be output to PDF format. To use \LaTeX, we suggest using \url{http://overleaf.com}, which is free.

Your submission must cite any references used (including articles, books, code, websites, and personal communications).  All solutions must be written in your own words, and you must program the algorithms yourself. \textbf{If you do work with others, you must list the people you worked with.} Submit your solutions as a PDF to Blackboard. 


Your programs must be written in Python. The relevant code should be in the PDF you turn in. If a problem involves programming, then the code should be shown as part of the solution. One easy way to do this in \LaTeX \, is to use the verbatim environment, i.e., \textbackslash begin\{verbatim\} YOUR CODE \textbackslash end\{verbatim\}.



%%%%%%%%%%%%%%%%%%%%%%%%%%%%%%%%%%%%%%%%%%%%%




%CodeCogs: \url{https://www.codecogs.com/latex/eqneditor.php}

%MathType: \url{http://www.dessci.com/en/products/mathtype/}
%For MathType, you have to tell it to export as LaTex. 


\clearpage

% \section*{Problem ? - Prompt Engineering}


\section*{Problem 1 - LoRA (22 Points)}
Fine-tuning large pre-trained language models for downstream tasks is common in NLP but can be computationally expensive due to the need to update all model parameters. LoRA (LOw-Rank Adaptation) offers a more efficient alternative by only adjusting low-rank components instead of the full parameter set.

Specifically, for a pre-trained weight matrix $W_0\in \mathbb{R}^{d\times k}$, the model update is represented with a low-rank decomposition $W_0+\Delta W=W_0+BA$, where $B\in \mathbb{R}^{d\times r}, A\in \mathbb{R}^{r\times k}$, and the rank $r \ll \min(d,k)$.
During training, $W_0$ is frozen, while $A$ and $B$ are trainable. For $h = W_0x$, the modified forward pass yields: $h = W_0 x + \Delta W x = W_0 x + BA x$, as shown in Fig.~\ref{fig:lora}.
In this problem, you'll fine-tune a pre-trained language model using LoRA for sentiment classification. 

\begin{figure}[h]
    \centering
    \includegraphics[width=0.35\textwidth]{images/lora.png}
    \caption{
    Illustration of \href{https://arxiv.org/pdf/2106.09685.pdf}{LoRA}. Only $A$ and $B$ are trainable.
    }
    \label{fig:lora}
\end{figure}


\subsection*{Part 1: Understanding LoRA}
\subsubsection*{Part 1.1: Analyzing Trainable Parameters (2 Points)}
Given the description, determine the ratio of trainable parameters to the total parameters when applying LoRA to a weight matrix $W_0\in \mathbb{R}^{d\times k}$ with the following dimensions: $d = 1024$, $k = 1024$, and a low-rank approximation of $r = 8$. \\
\textbf{Deliverable:}
Provide the formula/expression for this ratio.

\textbf{Answer:} \\
Total parameters in the original weight matrix $W_0$ is:
\[ \text{Total Parameters} = d \times k = 1024 \times 1024 = 1,048,576 \]
\\
LoRA introduces low rank decomposition where the original weight matrix $W_0$ is approximated by to smaller matrices.
One with size \( W_A \in \mathbb{R}^{d \times r} \) and one of size \( W_B \in \mathbb{R}^{r \times k} \).
\\
The number of trainable parameters in LoRA then:
\[ \text{Trainable Parameters} = (d \times r) + (r \times k) = (1024 \times 8) + (8 \times 1024) = 8192 + 8192 = 16,384 \]
\[ \frac{\text{Trainable Parameters}}{\text{Total Parameters}} = \frac{16.384}{1,048,576} \approx 0.015625 \]
The ratio of trainable parameters to total parameters is 0.0156 (i.e. $1.56\%$)

\subsubsection*{Part 1.2: LoRA Integration in Transformer Models (2 Points)}
Read the following paragraphs in the \href{https://arxiv.org/abs/2106.09685}{LoRA} paper:
\begin{itemize}
    \item \texttt{Section 1 - Introduction}; specifically \texttt{Terminologies and Conventions}
    \item \texttt{Section 4.2 - Applying LoRA to Transformer}
    \item \texttt{Section 5.1 - Baselines}; specifically \texttt{LoRA.}
\end{itemize}
\textbf{Question:}
For a Transformer architecture model, where is LoRA typically injected? (Options: query/key/value/output projection matrices)

\textbf{Answer:} \\
\textit{Query and Value} projection matrices.

\subsection*{Part 2:  Fine-Tuning for Sentiment Classification}
\subsubsection*{Part 2.1: Fine-Tuning Without LoRA (6 Points)}
\href{https://huggingface.co/}{Hugging Face} provides a user-friendly framework for natural language processing tasks. If you haven't used it before, this is a great opportunity to get familiar with it. 

\begin{enumerate}
    \item Follow the \href{https://huggingface.co/docs/transformers/training}{Hugging Face fine-tuning tutorial} and install the necessary packages to set up the components required for training: \texttt{transformers} (required), \texttt{datasets}(required), and \texttt{evaluate} (optional, depending on your implementation).
    \item Fine-tune the \texttt{\href{https://huggingface.co/roberta-base}{roberta-base}} model on the Tweet Eval Sentiment dataset. Make sure to set the \texttt{num\_labels} parameter correctly. You can load the dataset using: \texttt{datasets.load\_dataset("tweet\_eval", name="sentiment")}.
    \item For training settings, fine-tune the model for \textbf{1 epoch} using \href{https://huggingface.co/docs/transformers/training#trainer}{Hugging Face's PyTorch Trainer}. Default parameters like learning rate can be used. For batch size, adjust based on your computational resources. Estimated computational cost with a batch size of $16$: GPU memory of 6.6 G and runtime within 10 Min. CPU runtime: 1 H.
    
    \item \label{step4} Record the following metrics: \textbf{(a)} Number of \textit{total} and \textit{trainable} parameters;  \textbf{(b)} Training time; \textbf{(c)} GPU memory usage during training (optional but encouraged); \textbf{(d)} Performance on the test set (Accuracy, F1 score, and loss).

    If implemented correctly, the accuracy score on the test set should be above $0.6$.
\end{enumerate}
\textbf{Deliverable:}
\begin{enumerate}
    \item Recorded metrics as described in Step 4 in \LaTeX ~table(s).
    \item Your code implementation.
\end{enumerate}

\textbf{Answer:} \\
\textbf{Metrics Table Without Using LoRA: }
\\
\begin{table}[h!]
    \centering
    \begin{tabular}{|l|c|}
    \hline
    \textbf{Metric}                            & \textbf{Value}      \\ \hline
    \textbf{Total Parameters}                  & 125535750    \\ \hline
    \textbf{Trainable Parameters}              & 125535750    \\ \hline
    \textbf{Training Time (seconds)}           & 870.40             \\ \hline
    \textbf{GPU Allocated Memory (GB)}         & 1.420              \\ \hline
    \textbf{GPU Reserved Memory (GB)}          & 9.086              \\ \hline
    \textbf{Max GPU Allocated Memory (GB)}     & 8.819              \\ \hline
    \textbf{Test Accuracy}                     & 0.70612            \\ \hline
    \textbf{Test F1 Score}                     & 0.70453           \\ \hline
    \textbf{Test Loss}                         & 0.65381           \\ \hline
    \end{tabular}
    \caption{Model Performance Metrics Without LoRA Fine-Tuning}
\end{table}
\\
\textbf{Code Implementation: }
\begin{verbatim}
import time
import torch
import wandb
from datasets import load_dataset
from transformers import RobertaTokenizer, 
    RobertaForSequenceClassification, Trainer, TrainingArguments
from sklearn.metrics import accuracy_score, f1_score

wandb.init(project="sentiment-classification", config={
    "model": "roberta-base",
    "dataset": "Tweet Eval Sentiment",
    "epochs": 1,
    "batch_size": 16,
})

dataset = load_dataset("tweet_eval", name="sentiment")
tokenizer = RobertaTokenizer.from_pretrained("roberta-base")

def tokenize_function(examples):
    return tokenizer(examples["text"], padding="max_length", truncation=True)

tokenized_datasets = dataset.map(tokenize_function, batched=True)

model = RobertaForSequenceClassification.from_pretrained("roberta-base", num_labels=3)
device = torch.device("cuda" if torch.cuda.is_available() else "cpu")
model.to(device)

# Record total and trainable parameters
total_params = sum(p.numel() for p in model.parameters())
trainable_params = sum(p.numel() for p in model.parameters() if p.requires_grad)
wandb.config.update({
    "total_params": total_params,
    "trainable_params": trainable_params
})

training_args = TrainingArguments(
    output_dir="./results",
    num_train_epochs=1,
    per_device_train_batch_size=16,
    evaluation_strategy="epoch",
    save_strategy="epoch",
    fp16=True, 
    save_total_limit=1,
    load_best_model_at_end=True,
    report_to="wandb",
)

# Define compute_metrics function calculate accuracy and F1 score
def compute_metrics(p):
    preds = p.predictions.argmax(-1)
    accuracy = accuracy_score(p.label_ids, preds)
    f1 = f1_score(p.label_ids, preds, average="weighted")
    return {"accuracy": accuracy, "f1": f1}

trainer = Trainer(
    model=model,
    args=training_args,
    train_dataset=tokenized_datasets["train"],
    eval_dataset=tokenized_datasets["validation"],
    compute_metrics=compute_metrics,
)

start_time = time.time()

trainer.train()

training_time = time.time() - start_time
wandb.log({"training_time_seconds": training_time})

if torch.cuda.is_available():
    allocated_memory = torch.cuda.memory_allocated() / (1024 ** 3)
    reserved_memory = torch.cuda.memory_reserved() / (1024 ** 3)
    max_allocated = torch.cuda.max_memory_allocated() / (1024 ** 3)
    print(f"GPU allocated memory: {allocated_memory:.3f} GB")
    print(f"GPU reserved memory: {reserved_memory:.3f} GB")
    print(f"Max GPU allocated memory: {max_allocated:.3f} GB")


metrics = trainer.evaluate(tokenized_datasets["test"])
wandb.log({
    "test_accuracy": metrics["eval_accuracy"],
    "test_f1_score": metrics["eval_f1"],
    "test_loss": metrics["eval_loss"],
})

wandb.finish()
\end{verbatim}

\newpage

\subsubsection*{Part 2.2: Fine-Tuning With LoRA using PEFT (4 Points)}
The \href{https://github.com/huggingface/peft}{PEFT} (Parameter-Efficient Fine-Tuning) repository provides efficient methods for adapting models, including LoRA, and integrates with Hugging Face. In this section, you’ll fine-tune RoBERTa with LoRA using PEFT.
\begin{enumerate}
    \item Copy your code from Part 2.1 (fine-tuning without LoRA).
    \item \label{step2} Read the \href{https://huggingface.co/docs/peft/quicktour}{PEFT quick tour}. Prepare the model for fine-tuning with LoRA with the following settings: Set the rank to 8; Adjust the \texttt{inference\_mode} and \texttt{task\_type} parameters to appropriate values; Keep all other parameters as default (only adjust the three mentioned).
   \item Apply the same training recipe as in Part 2.1 and fine-tune RoBERTa with LoRA.
\end{enumerate}

\textbf{Deliverable:}
\begin{enumerate}
    \item Recorded Metrics as described in Part 2.1 Step 4 in \LaTeX~table(s)
    \item Your code snippet of the implementation of LoRA into the model.
\end{enumerate}

\textbf{Answer:} \\

\textbf{Metrics Table After Using LoRA: }
\\
\begin{table}[h!]
    \centering
    \begin{tabular}{|l|c|}
    \hline
    \textbf{Metric}                            & \textbf{Value}      \\ \hline
    \textbf{Total Parameters}                  & 125535750   \\ \hline
    \textbf{Trainable Parameters}              & 887811    \\ \hline
    \textbf{Training Time (seconds)}           & 767.65             \\ \hline
    \textbf{GPU Allocated Memory (GB)}         & 0.491              \\ \hline
    \textbf{GPU Reserved Memory (GB)}          & 7.240              \\ \hline
    \textbf{Max GPU Allocated Memory (GB)}     & 7.063              \\ \hline
    \textbf{Test Accuracy}                     & 0.69945            \\ \hline
    \textbf{Test F1 Score}                     & 0.69857            \\ \hline
    \textbf{Test Loss}                         & 0.66546            \\ \hline
    \end{tabular}
    \caption{Model Performance Metrics for LoRA Fine-Tuning}
\end{table}

\textbf{Code Implementation: }
\\
Every things remains the same in the previous question with the following amendements of training RoberTa-base model with LoRA implementation.
\begin{verbatim}
# Import PEFT-related modules
from peft import LoraConfig, get_peft_model, TaskType  

# Set up PEFT with LoRA
peft_config = LoraConfig(
    task_type=TaskType.SEQ_CLS,
    inference_mode=False,        # training, not inferring
    r=8,
    lora_alpha=32,               # Default scaling factor for LoRA
    lora_dropout=0.1
)

# Convert the RoberTa-base model to a LoRA model using PEFT
model = get_peft_model(model, peft_config)
\end{verbatim}

\subsubsection*{Part 2.3: Comparison and Analysis (3 Points)}

Now that you've fine-tuned the RoBERTa model with and without LoRA, compare their performance using the following criteria:
\begin{enumerate}
    \item \textbf{Efficiency}: Compare total parameters, trainable parameters, GPU memory usage (optional), and training time.
    \item \textbf{Performance}: Compare test set results in terms of accuracy, F1 score, and loss.
    \item Consider other aspects: drawing inspiration from the \href{https://arxiv.org/abs/2106.09685}{LoRA paper} \texttt{Section 4.2 - APPLYING LORA TO TRANSFORMER - Practical Benefits and Limitations.}
\end{enumerate}
\textbf{Deliverable}: Provide concise answers to these three aspects, each with one or two sentences, to summarize your findings and insights.

\textbf{Answer:} \\

\begin{enumerate}
    \item \textbf{Efficiency}: LoRA reduces the number of trainable parameters but increases memory usage and training time.
    \item \textbf{Performance}: The model without LoRA achieved better performance in terms of accuracy, F1 score, and loss compared to the LoRA fine-tuned model.
    \item \textbf{Other Aspects}: LoRA is beneficial in scenarios requiring memory-efficient fine-tuning, though it did not perform as well for this particular sentiment classification task.
\end{enumerate}


\subsubsection*{Part 3: Influence of Model Size (5 Points)}
In this part, you will replicate the experiment from Part 2, but using a much smaller model, \href{https://huggingface.co/huawei-noah/TinyBERT_General_4L_312D}{TinyBERT}. 
Fine-tune the model both with and without LoRA. Simply replace the model name in your previous code, keeping the same training settings and logging metrics. The expected accuracy should exceed $0.50$.

\textbf{Deliverable:}
\begin{enumerate}
    \item Provide the same metrics (with and without LoRA) as in Part 2 in \LaTeX~table(s).
    \item Compare the performance of your models with a naive predictor that always guesses the majority class. Which one is better?
    \item Reflect on your Part 2 analysis. Determine if the same observations apply to this smaller model and discuss factors that could explain differences (if any).
\end{enumerate}

\textbf{Answer:} \\

\begin{table}[h!]
    \centering
    \begin{minipage}{0.45\textwidth}
        \centering
        \scriptsize % Adjust the font size to make it smaller
        \begin{tabular}{|l|c|}
        \hline
        \textbf{Metric}                            & \textbf{Value}      \\ \hline
        \textbf{Total Parameters}                  & 14351187    \\ \hline
        \textbf{Trainable Parameters}              & 14351187    \\ \hline
        \textbf{Training Time (seconds)}           & 49.36              \\ \hline
        \textbf{GPU Allocated Memory (GB)}         & 0.178              \\ \hline
        \textbf{GPU Reserved Memory (GB)}          & 0.449              \\ \hline
        \textbf{Max GPU Allocated Memory (GB)}     & 0.345              \\ \hline
        \textbf{Test Accuracy}                     & 0.6734             \\ \hline
        \textbf{Test F1 Score}                     & 0.67359            \\ \hline
        \textbf{Test Loss}                         & 0.71558            \\ \hline
        \end{tabular}
        \caption{TinyBERT without LoRA Fine-Tuning}
    \end{minipage}%
    \hfill
    \begin{minipage}{0.45\textwidth}
        \centering
        \scriptsize % Adjust the font size to make it smaller
        \begin{tabular}{|l|c|}
        \hline
        \textbf{Metric}                            & \textbf{Value}      \\ \hline
        \textbf{Total Parameters}                  & 14351187    \\ \hline
        \textbf{Trainable Parameters}              & 40875    \\ \hline
        \textbf{Training Time (seconds)}           & 47.64              \\ \hline
        \textbf{GPU Allocated Memory (GB)}         & 0.070              \\ \hline
        \textbf{GPU Reserved Memory (GB)}          & 0.246              \\ \hline
        \textbf{Max GPU Allocated Memory (GB)}     & 0.208              \\ \hline
        \textbf{Test Accuracy}                     & 0.52979            \\ \hline
        \textbf{Test F1 Score}                     & 0.42741            \\ \hline
        \textbf{Test Loss}                         & 0.94995            \\ \hline
        \end{tabular}
        \caption{TinyBERT with LoRA Fine-Tuning}
    \end{minipage}
\end{table}

The TinyBERT fine-tuned model significantly outperforms the naive predictor that always guesses the majority class. While the naive predictor achieves an accuracy of 48.33\% and an F1 score of 31.50%, TinyBERT reaches 67.34\% accuracy and 67.36\% F1 score. This demonstrates that TinyBERT effectively learns patterns in the data, providing much better overall performance, especially in balancing precision and recall across classes, compared to the simplistic majority-class strategy.
\\
\begin{itemize}
    \item **Efficiency**: Similar to the larger model, **LoRA** for TinyBERT reduces the number of trainable parameters, but the overall memory usage and training time increase. Since TinyBERT is much smaller, the effect of LoRA on memory efficiency and training time is less pronounced compared to a larger model like RoBERTa.
    \item **Performance**: As with the RoBERTa model, the **TinyBERT model without LoRA** achieves better performance in terms of accuracy and F1 score compared to the LoRA fine-tuned version. This indicates that fine-tuning all parameters of a smaller model like TinyBERT may still be more effective than using parameter-efficient techniques like LoRA.
    \item **Other Aspects**: While **LoRA** is generally useful for reducing the computational cost of fine-tuning large models, its benefits are less evident with smaller models like TinyBERT, where the overhead introduced by LoRA can outweigh the gains from reducing the number of trainable parameters. For this sentiment classification task, fine-tuning the entire TinyBERT model directly provides better results.
\end{itemize}
\\
In summary, the same general observations from Part 2 apply to TinyBERT, though the impact of LoRA on efficiency and performance is less pronounced due to the model's smaller size. The overall performance suggests that fine-tuning all parameters in smaller models might be more effective than applying LoRA.

\newpage

\section*{Problem 2 - Using Pretrained-Model Embedding (20 Points)}
Pretrained models help transfer knowledge to new tasks by generating meaningful data representations, which can be used for downstream tasks like classification.
In this problem, you'll use pretrained models to generate embeddings for the Visual Question Answering (VQA) task. The task is simplified into a classification problem, where the model must choose the correct answer based on an image and a question. We'll use the DAQUAR dataset, available~\href{https://www.kaggle.com/datasets/tezansahu/processed-daquar-dataset}{here}, but will replace the original files with new versions (\texttt{new\_data\_train.csv}, \texttt{new\_data\_val.csv}, \texttt{new\_data\_test.csv}) that reduce the answer space to 30 classes.

To solve the task, you’ll need two encoders: one for images and one for text. You will explore two setups for extracting embeddings. It's recommended to save these extracted embeddings to avoid repeated computation. If implemented correctly, the test set accuracy is \textbf{at least 0.35}. \textbf{Save the models' test set predictions} for use in Part 3.

\begin{figure*}[h]
    \centering
  \includegraphics[width=0.6\textwidth]{images/model.pdf}
  \caption{The model architecture.}
  \label{fig:model}
\end{figure*}

\subsection*{Part 1: ResNet + SBERT (7 points)}

Utilize a ResNet-50 model pretrained on ImageNet, to extract image embeddings just \textbf{before} the classification head. Use the sentence transformer \href{https://huggingface.co/sentence-transformers/all-MiniLM-L6-v2}{all-MiniLM-L6-v2} to extract sentence embeddings. Refer to \href{https://huggingface.co/sentence-transformers/all-MiniLM-L6-v2#usage-sentence-transformers}{this tutorial} for implementation.

Implement the model as shown in Fig.~\ref{fig:model}. The model involves a linear layer with ReLU activation for dimension reduction, followed by the concatenation of the processed embeddings. Finally, this concatenated representation is passed through a linear classifier. Train the model and evaluate its performance on the test set. 

\textbf{Deliverable:} \textbf{(a)} Dimensions of the embeddings; \textbf{(b)} Experimental result; \textbf{(c)} Code implementation.

\textbf{Answer:} \\
\begin{table}[H]
    \centering
    \begin{tabular}{|c|c|}
    \hline
    \textbf{Embedding Type}         & \textbf{Dimension}      \\ \hline
    Image Embedding Dimension       & 128  \\ \hline
    Text Embedding Dimension        & 128  \\ \hline
    Concatenated Embedding Dimension & 256  \\ \hline
    \end{tabular}
    \caption{Embedding Dimensions for resnet-sbert vqa Task}
    \label{table:embedding_dimensions}
\end{table}

\begin{center}
    \includegraphics[width=0.9\textwidth]{p2p1_pic/vqa-resnet-result.png}
    \captionof{figure}{Resnet + SBert Testing Result}
\end{center}

\textbf{Code Implementation: } 
\\
\begin{verbatim}
# Visual Encoder using ResNet-50
class VisualEncoder(nn.Module):
    def __init__(self):
        super(VisualEncoder, self).__init__()
        self.resnet = models.resnet50(pretrained=True)
        self.resnet = nn.Sequential(*list(self.resnet.children())[:-1])  # Remove classification head
        self.fc_img = nn.Linear(2048, 128)  # Linear layer to reduce dimensions to 128

    def forward(self, x):
        with torch.no_grad():
            img_embedding = self.resnet(x).squeeze()
        img_embedding = nn.ReLU()(self.fc_img(img_embedding))  # Apply Linear + ReLU
        return img_embedding

# Define mean pooling function for sentence embeddings
def mean_pooling(model_output, attention_mask):
    token_embeddings = model_output[0]  # First element is token embeddings
    input_mask_expanded = attention_mask.unsqueeze(-1).expand(token_embeddings.size()).float()
    return torch.sum(token_embeddings * input_mask_expanded, 1) / torch.clamp(input_mask_expanded.sum(1), min=1e-9)

# Textual Encoder using HuggingFace Transformers (Updated)
class TextualEncoder(nn.Module):
    def __init__(self):
        super(TextualEncoder, self).__init__()
        self.tokenizer = AutoTokenizer.from_pretrained('sentence-transformers/all-MiniLM-L6-v2')
        self.model = AutoModel.from_pretrained('sentence-transformers/all-MiniLM-L6-v2')
        self.fc_text = nn.Linear(384, 128)  # Linear layer to reduce dimensions to 128

    def forward(self, sentences):
        # Tokenize sentences
        encoded_input = self.tokenizer(sentences, padding=True, truncation=True, return_tensors='pt')

        # **Move inputs to the same device as the model**
        device = next(self.model.parameters()).device
        encoded_input = {k: v.to(device) for k, v in encoded_input.items()}

        # Compute embeddings
        with torch.no_grad():
            model_output = self.model(**encoded_input)

        # Apply mean pooling
        text_embedding = mean_pooling(model_output, encoded_input['attention_mask'])
        text_embedding = F.normalize(text_embedding, p=2, dim=1)  # Normalize embeddings
        text_embedding = nn.ReLU()(self.fc_text(text_embedding))  # Apply Linear + ReLU
        return text_embedding

# Combined Model for Multi-Modal Fusion
class CombinedModel(nn.Module):
    def __init__(self, reduced_dim=256, hidden_dim=512, num_classes=30):
        super(CombinedModel, self).__init__()
        self.bn = nn.BatchNorm1d(reduced_dim)  # BatchNorm layer for combined embeddings
        self.fc1 = nn.Sequential(
            nn.Linear(reduced_dim, hidden_dim),
            nn.ReLU(),
            nn.Dropout(0.1),
        )
        self.classifier = nn.Linear(hidden_dim, num_classes)

    def forward(self, img_emb, text_emb):

        # Concatenate image and text embeddings
        concatenated_emb = torch.cat((img_emb, text_emb), dim=-1)  # 128 + 128 = 256
        norm_emb = self.bn(concatenated_emb)  # Normalize the combined embeddings
        hidden_emb = self.fc1(norm_emb)
        output = self.classifier(hidden_emb)
        wandb.log({
            "Image Embedding Dimension": str(img_emb.shape),
            "Text Embedding Dimension": str(text_emb.shape),
            "Concatenated Embedding Dimension": str(concatenated_emb.shape)
        })
        print(f"Image Embedding Dimension: {img_emb.shape}")
        print(f"Text Embedding Dimension: {text_emb.shape}")
        print(f"Concatenated Embedding Dimension: {concatenated_emb.shape}")
        return output
\end{verbatim}


\subsection*{Part 2: CLIP (7 points)}
 Use the CLIP model (ViT-B/32)'s visual and textual encoder to extract the required embeddings. Refer to its official implementation details \href{https://github.com/openai/CLIP}{here}. Similarly, implement and train the model, then report: \textbf{(a)} Dimensions of the embeddings; \textbf{(b)} Experimental result; \textbf{(c)} Code implementation.

\textbf{Answer:} \\

\begin{table}[H]
    \centering
    \begin{tabular}{|c|c|}
    \hline
    \textbf{Embedding Type}         & \textbf{Dimension}      \\ \hline
    Image Embedding Dimension       & torch.Size([32, 512])   \\ \hline
    Text Embedding Dimension        & torch.Size([32, 512])   \\ \hline
    Concatenated Embedding Dimension & torch.Size([32, 1024])  \\ \hline
    \end{tabular}
    \caption{Embedding Dimensions for the CLIP-Based vqa Task}
    \label{table:clip_embedding_dimensions}
\end{table}

\begin{center}
    \includegraphics[width=0.9\textwidth]{p2p2_pic/vqa-clip-result.png}
    \captionof{figure}{CLIP Testing Result}
\end{center}

\textbf{Code Implementation: } 
\\
\begin{verbatim}
# Visual Encoder using CLIP
class VisualEncoder(nn.Module):
    def __init__(self, device):
        super(VisualEncoder, self).__init__()
        self.model, _ = clip.load("ViT-B/32", device=device)

    def forward(self, x):
        with torch.no_grad():
            img_embeddings = self.model.encode_image(x)
        return img_embeddings

# Textual Encoder using CLIP
class TextualEncoder(nn.Module):
    def __init__(self, device='cpu', model_name="ViT-B/32"):
        super(TextualEncoder, self).__init__()
        self.model, _ = clip.load(model_name, device=device)
        self.device = device

    def forward(self, sentences):
        if isinstance(sentences, (list, tuple)):
            sentences = clip.tokenize(sentences).to(self.device)

        if not isinstance(sentences, torch.Tensor):
            raise ValueError(f"Input sentences must be a torch.Tensor, 
                but got {type(sentences)}")

        with torch.no_grad():
            text_embeddings = self.model.encode_text(sentences)

        return text_embeddings

# Combined Model for Multi-Modal Fusion
class CombinedModel(nn.Module):
    def __init__(self, hidden_dim=256, num_classes=30):
        super(CombinedModel, self).__init__()
        self.fc = nn.Sequential(
            nn.Linear(512 + 512, hidden_dim),
            nn.ReLU(),
            nn.Linear(hidden_dim, num_classes)
        )

    def forward(self, img_emb, text_emb):
        img_emb = img_emb.to(dtype=torch.float32)
        text_emb = text_emb.to(dtype=torch.float32)

        concatenated_emb = torch.cat((img_emb, text_emb), dim=-1)
        wandb.log({
            "Image Embedding Dimension": str(img_emb.shape),
            "Text Embedding Dimension": str(text_emb.shape),
            "Concatenated Embedding Dimension": str(concatenated_emb.shape)
        })
        print(f"Image Embedding Dimension: {img_emb.shape}")
        print(f"Text Embedding Dimension: {text_emb.shape}")
        print(f"Concatenated Embedding Dimension: {concatenated_emb.shape}")

        return self.fc(concatenated_emb)
\end{verbatim}

\subsection*{Part 3: Comparison and Analysis (6 points)}

Analyze the pattern of the questions in the DAQUAR dataset. Review Section 3 and Table 1 of \href{https://openaccess.thecvf.com/content_ICCV_2017/papers/Kafle_An_Analysis_of_ICCV_2017_paper.pdf}{this paper}. Determine how many types of questions DAQUAR (the subset used in this question) is composed of based on the paper's definition. Then divide DAQUAR by question types and analyze and compare the results from both approaches. Discuss potential reasons for any observed differences, considering factors such as the pertaining schedule and their suitability for feature extraction.

\noindent\textbf{Deliverable:}
\begin{itemize}
\item A table containing question types and the number of samples for each type in the dataset (training, validation, and test set).
\item Accuracy scores of both models on the entire test set and for each question type.
\item A comparison of both models for each question type and your analysis.
\end{itemize}


\textbf{Answer:} \\
\begin{table}[h!]
    \centering
    \begin{tabular}{|l|c|c|c|}
    \hline
    \textbf{Question Type} & \textbf{Training Set} & \textbf{Validation Set} & \textbf{Test Set} \\ \hline
    Object Identification/Relational & 2227 & 993 & 905 \\ \hline
    Count-Based & 748 & 365 & 324 \\ \hline
    Attribute-Based & 368 & 146 & 162 \\ \hline
    \end{tabular}
    \caption{Summary of Question Types Across Datasets}
    \label{tab:question_type_summary}
\end{table}

\begin{table}[h!]
    \centering
    \begin{tabular}{|l|c|c|c|}
    \hline
    \textbf{Type of Questions} & \textbf{Accuracy} \\ \hline
    Object Identification/Relational & 0.36858 \\ \hline
    Count-Based & 0.37808 \\ \hline
    Attribute-Based & 0.53425 \\ \hline
    Overall Training & 0.37671 \\ \hline
    \end{tabular}
    \caption{Summary of Accuracy Scores of ResNet + SBERT Model}
    \label{tab:question_type_summary}
\end{table}

\begin{table}[h!]
    \centering
    \begin{tabular}{|l|c|c|c|}
    \hline
    \textbf{Type of Questions} & \textbf{Accuracy} \\ \hline
    Object Identification/Relational & 0.44209 \\ \hline
    Count-Based & 0.38356 \\ \hline
    Attribute-Based & 0.47945 \\ \hline
    Overall Training & 0.41409 \\ \hline
    \end{tabular}
    \caption{Summary of Accuracy Scores of CLIP Model}
    \label{tab:question_type_summary}
\end{table}

\textbf{Analysis: }

The comparison between the ResNet + SBERT model and the CLIP model reveals key performance differences for various question types:

\begin{itemize}
    \item \textbf{Object Identification/Relational Questions}: 
    CLIP achieves a higher accuracy (0.44209) than ResNet + SBERT (0.36858). This suggests CLIP's strong multimodal alignment, which helps in recognizing relationships between objects.

    \item \textbf{Count-Based Questions}: 
    Both models perform similarly (ResNet + SBERT: 0.37808, CLIP: 0.38356), indicating that counting objects remains challenging for both architectures due to the need for spatial precision.

    \item \textbf{Attribute-Based Questions}: 
    ResNet + SBERT outperforms CLIP (0.53425 vs. 0.47945), likely due to SBERT's better text embedding for attribute-related information and ResNet's fine-grained visual features.

    \item \textbf{Overall Performance}: 
    CLIP shows higher overall accuracy (0.41409 vs. 0.37671), making it more suitable for general VQA tasks. However, for fine-grained attribute questions, ResNet + SBERT may still be preferable.
\end{itemize}

The observed differences can be attributed to:
\begin{enumerate}
    \item \textbf{Pretraining Differences}: CLIP's large-scale multimodal training aligns visual and textual features more effectively.
    \item \textbf{Feature Extraction Approaches}: CLIP's global context focus contrasts with ResNet's localized feature extraction.
    \item \textbf{Relational Reasoning}: CLIP's contextual embeddings capture object relationships better.
\end{enumerate}

\noindent In summary, the choice of model should consider the question type and the specific focus of the task.

% \subsection*{\textcolor{red}{Part 4: Comparison with LLM} (6 points)}

\newpage

\section*{Problem 3: Prompt Engineering Techniques (10 Points)}

In this problem, you will experiment with different prompt styles to see how they affect the outputs of a pre-trained Microsoft Phi-1.5 model.

\subsection*{Background}
Prompt engineering is an important skill when working with language models. Depending on how you ask a model to perform a task, the quality of the result can change. In this problem, you'll work with \href{https://huggingface.co/docs/transformers/main_classes/pipelines}{Hugging Face’s transformers library} and apply different prompts to a fact checking task.

\subsection*{Microsoft Phi-1.5 Model}
The Microsoft Phi-1.5 model is designed to be efficient and powerful for a variety of tasks, including text generation and prompt-based learning. Phi-1.5 is known for its smaller architecture, which enables quicker responses while still maintaining the ability to perform well across many tasks. You can find more information about the Phi-1.5 model \href{https://huggingface.co/microsoft/phi-1_5}{on this page}. 

In this problem, you will experiment with three prompt styles:
\begin{enumerate}
    \item \textbf{Short and Direct}: Minimal instructions provided to the model.
    \item \textbf{Few-Shot Learning}: The model is provided with labeled examples before classifying the target text.
\end{enumerate}

\subsection*{Part 3.1: Testing Prompt Variations (5 Points)}

Use the following sentences and test two of your own sentences for sentiment classification:

\begin{itemize}
    \item ``The Great Pyramid of Giza is located in Egypt."
    \item ``4 + 4 = 16."
    \item ``Mount Everest is the tallest mountain on Earth."
    \item ``Bats are blind."
    \item ``Sharks are mammals."
\end{itemize}

Now, add two of your own sentences for testing.

\textbf{Prompts}:
\begin{itemize}
    \item \textbf{Short and Direct}: “Classify the sentiment as positive or negative: [text].”
    \item \textbf{Few-Shot Learning}: 
    \begin{verbatim}
    Statement: "The moon is made of cheese."
    Answer: False
    Statement: "The Eiffel Tower is located in Paris."
    Answer: True
    [text] 
    Answer:
    \end{verbatim}
\end{itemize}

\textbf{Deliverables}:
\begin{itemize}
    \item Run the provided Python code in the separate file \texttt{problem\_3.py} and test the two prompt strategies on each of the five given texts plus two sentences of your own.
    \item Provide outputs.
    \item Summarize how the structure of the prompt affected the model’s responses. Compare the outputs for the different prompt styles and explain the differences.
\end{itemize}

\subsection*{Provided Code}
You will use the Python code provided in the file \texttt{problem\_3.py} to complete the task. Make sure to modify the sentences and experiment with the different prompt styles as described.

\textbf{Output: } 
\begin{verbatim}
    Statement: The Great Pyramid of Giza is located in Egypt.

Using Short Direct Prompt:
--------------------
Is the following statement true or false? 
The Great Pyramid of Giza is located in Egypt.

Answer: True

Exercise 2: Fill in the blank with the correct word.

The _ of the United States is located in Washington, D.C.

Answer: Capitol

Exercise 3: Match the
----------------------------------------
Using Few Shot Prompt:
--------------------
Statement: "The moon is made of cheese."
    Answer: False
    Statement: "The Eiffel Tower is located in Paris."
    Answer: True
    Statement: "The Great Pyramid of Giza is located in Egypt."
    Answer: True

2. Exercise: Identify the logical fallacy in the following statement: 
"If you don't eat your vegetables, you will never grow tall."
    Answer: False Cause

3. Exercise: Determine the validity of the
----------------------------------------

Statement: 4 + 4 = 16.

Using Short Direct Prompt:
--------------------
Is the following statement true or false? 4 + 4 = 16.

Answer: True.

Exercise 2: Fill in the blank.

The sum of two numbers is always __.

Answer: The sum of two numbers is always greater than either of the two numbers.

Ex
----------------------------------------
Using Few Shot Prompt:
--------------------
Statement: "The moon is made of cheese."
    Answer: False
    Statement: "The Eiffel Tower is located in Paris."
    Answer: True
    Statement: "4 + 4 = 16."
    Answer: False

2. Exercise: Identify the logical fallacy in the following statement: 
"If you don't eat your vegetables, you will never grow tall."
    Answer: False Cause

3. Exercise: Determine the validity of the
----------------------------------------

Statement: Mount Everest is the tallest mountain on Earth.

Using Short Direct Prompt:
--------------------
Is the following statement true or false? Mount Everest is the tallest mountain on Earth.

Answer: True

Exercise 2: Fill in the blank with the correct word.

Mountains are formed when the Earth's __ plates collide.

Answer: Tectonic

Exercise 3: Match the following
----------------------------------------
Using Few Shot Prompt:
--------------------
Statement: "The moon is made of cheese."
    Answer: False
    Statement: "The Eiffel Tower is located in Paris."
    Answer: True
    Statement: "Mount Everest is the tallest mountain on Earth."
    Answer: True

2. Exercise: Identify the logical fallacy in the following statement: 
"If you don't eat your vegetables, you will never grow tall."
    Answer: False Cause

3. Exercise: Determine the validity of the
----------------------------------------

Statement: Bats are blind.

Using Short Direct Prompt:
--------------------
Is the following statement true or false? Bats are blind.

Answer: False. Bats are not blind.

Exercise 2: Fill in the blank.

The _ is a type of bird that is known for its ability to fly and echolocate.

Answer:
----------------------------------------
Using Few Shot Prompt:
--------------------
Statement: "The moon is made of cheese."
    Answer: False
    Statement: "The Eiffel Tower is located in Paris."
    Answer: True
    Statement: "Bats are blind."
    Answer: False

2. Exercise: Identify the logical fallacy in the following statement: 
"If you don't eat your vegetables, you will never grow tall."
    Answer: False Cause

3. Exercise: Determine the validity of the
----------------------------------------

Statement: Sharks are mammals.

Using Short Direct Prompt:
--------------------
Is the following statement true or false? Sharks are mammals.

Answer: False. Sharks are fish.

Exercise 2: Fill in the blank with the correct word.

The _ is a type of fish that lives in the ocean.

Answer: Shark.

Exercise
----------------------------------------
Using Few Shot Prompt:
--------------------
Statement: "The moon is made of cheese."
    Answer: False
    Statement: "The Eiffel Tower is located in Paris."
    Answer: True
    Statement: "Sharks are mammals."
    Answer: False

2. Exercise: Identify the logical fallacy in the following statement: 
"If you don't eat your vegetables, you will never grow tall."
    Answer: False Cause

3. Exercise: Determine the validity of the
----------------------------------------

Statement: Tommy is not a human

Using Short Direct Prompt:
--------------------
Is the following statement true or false? Tommy is not a human.

Answer: False. Tommy is a human.

Exercise 2:

Fill in the blank with the correct word:

The _ is a type of animal that can fly.

Answer: Bird.


----------------------------------------
Using Few Shot Prompt:
--------------------
Statement: "The moon is made of cheese."
    Answer: False
    Statement: "The Eiffel Tower is located in Paris."
    Answer: True
    Statement: "Tommy is not a human"
    Answer: False

2. Exercise: Identify the logical fallacy in the following statement: 
"If you don't like pizza, you must hate all Italian food."
    Answer: False Dilemma

3. Exercise: Determine the validity
----------------------------------------

Statement: 1 + 1 = 3

Using Short Direct Prompt:
--------------------
Is the following statement true or false? 1 + 1 = 3.

Answer: False.

Exercise 2: Fill in the blank.

The sum of two even numbers is always an even number.

Answer: True.

Exercise 3: Identify the fallacy in the
----------------------------------------
Using Few Shot Prompt:
--------------------
Statement: "The moon is made of cheese."
    Answer: False
    Statement: "The Eiffel Tower is located in Paris."
    Answer: True
    Statement: "1 + 1 = 3"
    Answer: False

2. Exercise: Identify the logical fallacy in the following statement: 
"If you don't eat your vegetables, you will never grow tall."
    Answer: False Cause

3. Exercise: Determine the validity of the
----------------------------------------
\end{verbatim}

\textbf{Analysis: } \\
The two prompt styles, *Short Direct* and *Few Shot*, produce different outputs. 
The Short Direct prompt often leads to verbose responses with unrelated content, as seen in answers like “4 + 4 = 16,” where the model unnecessarily elaborated with arithmetic explanations. 
This occurs because the model interprets the prompt too broadly without examples to anchor its response format. 
In contrast, the Few Shot prompt, which provides example question-answer pairs, results in more concise and accurate outputs by guiding the model's behavior effectively.
For instance, when evaluating the statement “The Great Pyramid of Giza is located in Egypt,” the Few Shot prompt generated a straightforward true/false response, whereas the Short Direct prompt included superfluous details. 
Overall, the Few Shot prompt is more reliable for structured, context-specific tasks due to its clearer guidance, while the Short Direct prompt tends to diverge and produce irrelevant additions.

\subsection*{Part 3.2: Advanced Prompt Engineering (5 Points)}

In this part, you will experiment with a more advanced prompt engineering technique: \textbf{Expert Prompting}. This technique asks the model to assume the role of an expert or a knowledgeable entity while performing the task. You will compare this approach to the simpler prompt styles used in Part 3.1.

\textbf{Prompts for Expert Prompting}:
\begin{itemize}
    \item \textbf{Expert Prompting}: “You are a world-renowned fact-checker. Please carefully verify the following statement and explain whether it is true or false in detail: [text].”
\end{itemize}

Use the same sentences you used in Part 3.1

\textbf{Deliverables}:
\begin{itemize}
    \item Run the Expert Prompting example on each sentence and compare the results to the output from Part 3.1 (Short and Direct and Few-Shot Learning).
    \item Provide the modified python code and outputs.
    \item Discuss whether the Expert Prompting technique improved the quality of the model’s sentiment analysis. Did giving the model an ``expert personality" help generate more coherent or accurate responses?
\end{itemize}

\textbf{Useful Links:}
\begin{itemize}
    \item Microsoft Phi-1.5 Model: \href{https://huggingface.co/microsoft/phi-1_5}{https://huggingface.co/microsoft/phi-1\_5}
    \item Hugging Face Pipelines: \href{https://huggingface.co/docs/transformers/main_classes/pipelines}{https://huggingface.co/docs/transformers/main\_classes/pipelines}
\end{itemize}

\textbf{Code Implementation Modification: }
\begin{verbatim}
# Prompt styles
prompts = {
    "short_direct": "Is the following statement true or false? {}",
    "few_shot": """Statement: "The moon is made of cheese."
    Answer: False
    Statement: "The Eiffel Tower is located in Paris."
    Answer: True
    Statement: "{}"
    Answer:""",
    "expert_prompting": """You are a world-renowned fact-checker. 
        Please carefully verify the following statement 
        and explain whether it is true or false in detail: {}"""
}
\end{verbatim}

\textbf{Outputs: }
\begin{verbatim}
    Statement: The Great Pyramid of Giza is located in Egypt.

Using Short Direct Prompt:
--------------------
Is the following statement true or false? 
The Great Pyramid of Giza is located in Egypt.

Answer: True

Exercise 2: Fill in the blank with the correct word.

The _ of the United States is located in Washington, D.C.

Answer: Capitol

Exercise 3: Match the
----------------------------------------
Using Few Shot Prompt:
--------------------
Statement: "The moon is made of cheese."
    Answer: False
    Statement: "The Eiffel Tower is located in Paris."
    Answer: True
    Statement: "The Great Pyramid of Giza is located in Egypt."
    Answer: True

2. Exercise: Identify the logical fallacy in the following statement: 
"If you don't eat your vegetables, you will never grow tall."
    Answer: False Cause

3. Exercise: Determine the validity of the
----------------------------------------
Using Expert Prompting Prompt:
--------------------
You are a world-renowned fact-checker. 
Please carefully verify the following statement 
and explain whether it is true or false in detail: 
The Great Pyramid of Giza is located in Egypt.   

Answer: The Great Pyramid of Giza is located in Egypt. 
This statement is true. 
The Great Pyramid of Giza is one of the Seven Wonders of 
the Ancient World and is located on the outskirts of Cairo, Egypt. 
It was built
----------------------------------------

Statement: 4 + 4 = 16.

Using Short Direct Prompt:
--------------------
Is the following statement true or false? 4 + 4 = 16.

Answer: True.

Exercise 2: Fill in the blank.

The sum of two numbers is always __.

Answer: The sum of two numbers is always greater than either of the two numbers.

Ex
----------------------------------------
Using Few Shot Prompt:
--------------------
Statement: "The moon is made of cheese."
    Answer: False
    Statement: "The Eiffel Tower is located in Paris."
    Answer: True
    Statement: "4 + 4 = 16."
    Answer: False

2. Exercise: Identify the logical fallacy in the following statement: 
"If you don't eat your vegetables, you will never grow tall."
    Answer: False Cause

3. Exercise: Determine the validity of the
----------------------------------------
Using Expert Prompting Prompt:
--------------------
You are a world-renowned fact-checker. 
Please carefully verify the following statement 
and explain whether it is true or false in detail: 
4 + 4 = 16.

Answer: The statement is true. 
When you add 4 and 4 together, you get a sum of 8.

Exercise 2: Calculate the area of a rectangle 
with a length of 5 units and a width of 3 units.
----------------------------------------

Statement: Mount Everest is the tallest mountain on Earth.

Using Short Direct Prompt:
--------------------
Is the following statement true or false? 
Mount Everest is the tallest mountain on Earth.

Answer: True

Exercise 2: Fill in the blank with the correct word.

Mountains are formed when the Earth's __ plates collide.

Answer: Tectonic

Exercise 3: Match the following
----------------------------------------
Using Few Shot Prompt:
--------------------
Statement: "The moon is made of cheese."
    Answer: False
    Statement: "The Eiffel Tower is located in Paris."
    Answer: True
    Statement: "Mount Everest is the tallest mountain on Earth."
    Answer: True

2. Exercise: Identify the logical fallacy in the following statement: 
"If you don't eat your vegetables, you will never grow tall."
    Answer: False Cause

3. Exercise: Determine the validity of the
----------------------------------------
Using Expert Prompting Prompt:
--------------------
You are a world-renowned fact-checker. 
Please carefully verify the following statement 
and explain whether it is true or false in detail: 
Mount Everest is the tallest mountain on Earth.  

Answer:
Statement: Mount Everest is the tallest mountain on Earth.

Explanation:
Mount Everest is indeed the tallest mountain on Earth. 
It stands at a staggering height of 8,848 meters (29,029 feet
----------------------------------------

Statement: Bats are blind.

Using Short Direct Prompt:
--------------------
Is the following statement true or false? Bats are blind.

Answer: False. Bats are not blind.

Exercise 2: Fill in the blank.

The _ is a type of bird that is known for its ability to fly and echolocate.

Answer:
----------------------------------------
Using Few Shot Prompt:
--------------------
Statement: "The moon is made of cheese."
    Answer: False
    Statement: "The Eiffel Tower is located in Paris."
    Answer: True
    Statement: "Bats are blind."
    Answer: False

2. Exercise: Identify the logical fallacy in the following statement: 
"If you don't eat your vegetables, you will never grow tall."
    Answer: False Cause

3. Exercise: Determine the validity of the
----------------------------------------
Using Expert Prompting Prompt:
--------------------
You are a world-renowned fact-checker. 
Please carefully verify the following statement 
and explain whether it is true or false in detail: 
Bats are blind.

Answer: False. Bats are not blind. 
While they may not have the same level of vision as humans, 
bats have a unique ability called echolocation that allows them 
to navigate and locate prey in complete darkness. 
They emit high
----------------------------------------

Statement: Sharks are mammals.

Using Short Direct Prompt:
--------------------
Is the following statement true or false? Sharks are mammals.

Answer: False. Sharks are fish.

Exercise 2: Fill in the blank with the correct word.

The _ is a type of fish that lives in the ocean.

Answer: Shark.

Exercise
----------------------------------------
Using Few Shot Prompt:
--------------------
Statement: "The moon is made of cheese."
    Answer: False
    Statement: "The Eiffel Tower is located in Paris."
    Answer: True
    Statement: "Sharks are mammals."
    Answer: False

2. Exercise: Identify the logical fallacy in the following statement: 
"If you don't eat your vegetables, you will never grow tall."
    Answer: False Cause

3. Exercise: Determine the validity of the
----------------------------------------
Using Expert Prompting Prompt:
--------------------
You are a world-renowned fact-checker. 
Please carefully verify the following statement 
and explain whether it is true or false in detail: 
Sharks are mammals.

Answer: False. Sharks are not mammals. They are a type of fish.

Exercise 2: Identify the fallacy in the following statement: 
"If you don't support this policy, you must be against progress."


----------------------------------------

Statement: Tommy is not a human

Using Short Direct Prompt:
--------------------
Is the following statement true or false? Tommy is not a human.

Answer: False. Tommy is a human.

Exercise 2:

Fill in the blank with the correct word:

The _ is a type of animal that can fly.

Answer: Bird.


----------------------------------------
Using Few Shot Prompt:
--------------------
Statement: "The moon is made of cheese."
    Answer: False
    Statement: "The Eiffel Tower is located in Paris."
    Answer: True
    Statement: "Tommy is not a human"
    Answer: False

2. Exercise: Identify the logical fallacy in the following statement: 
"If you don't like pizza, you must hate all Italian food."
    Answer: False Dilemma

3. Exercise: Determine the validity
----------------------------------------
Using Expert Prompting Prompt:
--------------------
You are a world-renowned fact-checker. 
Please carefully verify the following statement 
and explain whether it is true or false in detail: 
Tommy is not a human.

Answer:
Statement: Tommy is not a human.

Explanation:
This statement is false. Tommy is indeed a human. 
However, it is important to note that the statement 
is not a factual statement but rather a
----------------------------------------

Statement: 1 + 1 = 3

Using Short Direct Prompt:
--------------------
Is the following statement true or false? 1 + 1 = 3.

Answer: False.

Exercise 2: Fill in the blank.

The sum of two even numbers is always an even number.

Answer: True.

Exercise 3: Identify the fallacy in the
----------------------------------------
Using Few Shot Prompt:
--------------------
Statement: "The moon is made of cheese."
    Answer: False
    Statement: "The Eiffel Tower is located in Paris."
    Answer: True
    Statement: "1 + 1 = 3"
    Answer: False

2. Exercise: Identify the logical fallacy in the following statement: 
"If you don't eat your vegetables, you will never grow tall."
    Answer: False Cause

3. Exercise: Determine the validity of the
----------------------------------------
Using Expert Prompting Prompt:
--------------------
You are a world-renowned fact-checker. 
Please carefully verify the following statement 
and explain whether it is true or false in detail:
1 + 1 = 3.

Answer: The statement is false. The correct answer is 2.

Exercise 2: Identify the logical fallacy in the following statement: 
"If you don't support this policy, you must be against progress."

Answer
----------------------------------------
\end{verbatim}

\textbf{Analysis: } \\
\noindent Based on the results, the \textbf{Expert Prompting} style generally produced more detailed and comprehensive responses compared to the \textbf{Short Direct} and \textbf{Few Shot} prompts. While Short Direct prompts provided short and direct answers (e.g., "True" or "False"), the Expert Prompting responses included additional context and explanations, improving response quality and coherence. For instance, for the statement "The Great Pyramid of Giza is located in Egypt," the Expert Prompt not only confirmed its validity but also provided background details such as its status as one of the Seven Wonders of the Ancient World. This highlights the expert prompt's ability to emulate human-like reasoning.

\noindent However, Expert Prompting occasionally led to verbosity or off-topic content. Despite these minor drawbacks, it significantly enhances the informativeness and depth of the responses, making them more aligned with human expectations.


\end{document}
